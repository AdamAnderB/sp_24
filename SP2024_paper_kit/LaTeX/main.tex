\documentclass[a4paper]{article}

\usepackage{INTERSPEECH2021}
\usepackage{subfiles}
\usepackage{datatool}
\usepackage{fmtcount}
\usepackage{xstring}
\usepackage{substr} % Include this in your preamble
\DTLloaddb{cronbach}{scripts/data_output/cronbach_alpha.csv}
\DTLloaddb{modelsouts}{scripts/data_output/dynamic_models.csv}
\DTLloaddb{partrem}{scripts/data_output/participant_removal.csv}
\DTLloaddb{taskrem}{scripts/data_output/taskremoval.csv}

\newcommand{\livedata}[2]{%
    \DTLfetch{#1}{Statistic}{#2}{Value}%
}

\title{Measuring music and prosody: accounting for variation in non-native speech discrimination with L1, L2, music skills, and working memory}
%Paper submission must be anonymous. Only fill in author information for the final PDF.
\name{Author Anonymous$^1$, 
Co-author Anonymous$^1$,
Co-author Anonymous$^1$,
Co-author Anonymous$^1$}
%The maximum number of authors in the author list is twenty. If the number of contributing authors is more than twenty, they should be listed in a footnote or in acknowledgement section, as appropriate.
\address{
  $^1$Author Affiliation}
\email{author@university.edu, 
coauthor@company.com}

\begin{document}

\maketitle
% 
\begin{abstract}
The dynamics of non-native speech perception remain poorly understood, especially in accounting for specialized skills/training. One such skill, musical ability, has been shown to positively impact sensitivity to speech sounds, yet how musical ability is operationalized and measured varies from study to study. Individuals’ musical abilities vary in exposure-duration, skill type (e.g., voice, percussion), and skill-level. Here, we take an individual differences (n=38) approach to explore sensitivity in non-native speech discrimination of prosodic contrasts. We measure language background, general cognitive measures, and three measures of musical ability: auditory-motor temporal integration \cite{Kachlicka_Saito_Tierney_2019}, auditory discrimination \cite[MET;]{Wallentin_Nielsen_Friis-Olivarius_Vuust_Vuust_2010}), and musical sophistication \cite[Gold-MSI;]{Müllensiefen_Gingras_Musil_Stewart_2014}. We measured prosodic sensitivity using three AX discrimination tasks and signal detection measures (d'/c): Mandarin tone (primarily cued by pitch), Italian and Japanese (non-)geminates (primarily cued by duration). Results suggest music background, discrimination, and auditory-motor temporal integration capture related –yet divergent– aspects of music experience. Additionally, music sub-skills (e.g., pitch perception) have unequal contributions to non-native speech sensitivity across languages' respective linguistic cues (e.g., tone). Findings support models of non-native speech perception, which consider cognitive factors and auditory experience outside of language experience.

\end{abstract}
\noindent\textbf{Index Terms}:  Individual differences, Music, Non-native speech perception, Measuring prosody


\section{Introduction}
awesome text

This is gs alpha
\livedata{cronbach}{gs}

this is item removal for italian task:
\livedata{taskrem}{italian.remove}

this is participants removal for language task:
\livedata{taskrem}{particip_remove_lang.remove}

This is the mandarin model out:
\livedata{modelsouts}{Mandarin.(Intercept).estimate}

This is how many participants we removed lang removal:
\livedata{taskrem}{part_remove_lang.before}

This is how many participants we kept:
\livedata{partrem}{kept_participants}

This is how many participants we started with for the cmu data:
\livedata{partrem}{data_exp_141883-v12 .before}

This is how many participants we kept for the cmu data:
\livedata{partrem}{data_exp_141883-v12 .after}

\section{Outline}
\subfile{outline.tex}

\section{methods}

\subsubsection{participants}
 The recruitment of \livedata{partrem}{starting_participants} participants was managed through Prolific \cite{Palan_2018} (n=\livedata{partrem}{data_exp_142778-v2.before})  and in-person recruitment (n=\livedata{partrem}{data_exp_141883-v12.before}). Not included in this, 22 participants were rejected from participation due to failing initial requirements (i.e., eight removed for failed headphone-check \cite{milne_2021} and 14 removed for eye-tracking calibration failure, which is not reported here). Of the \livedata{partrem}{starting_participants} participants who participated in the study, \livedata{partrem}{removed_participants} were removed for low accuracy scores. Three median absolute deviations (MAD) was used as the standard for removal to maximize retained participants \cite{Leys_2013}. Of these, \livedata{taskrem}{particip_remove_lang.remove} were removed for being below MAD range in the language tasks and \livedata{taskrem}{rhythm_part.remove} removed for low performance in auditory-motor integration task.  After removal, \livedata{partrem}{kept_participants} participants \livedata{partrem}{part_mean} data were retained for analysis. 


As in Porretta et al. (2020), participants were asked
to estimate their total lifetime experience interacting with accented Chinese speakers
(as a percentage of their lifetime interactions). The resulting 49 participants’ reported
amount of Chinese accent experience (range= 0-100 M = 10.47% (SD = 14.13), was
right skewed. Following Porretta et al. (2020), log transformation (with a constant of 1)
and a prior scaling to 0-30 range was employed (range = 0-3.43, M = 0.99, SD = 0.92).

\subsection{Figures}

stuff here Figure~\ref{fig:raw_data}.

\begin{figure}[t]
  \centering
  \includegraphics[width=\linewidth]{SP_24_visuals/Normalized_score_distrubutions_by_task.pdf}
  \caption{Raw data}
  \label{fig:raw_data}
\end{figure}

\subsection{Results}

all of our fancy results

~\ref{fig:model}.

\begin{figure}[t]
  \centering
  \includegraphics[width=\linewidth]{SP_24_visuals/Japanese,Italian,_Mandarin_max_models_structure:_parsimonious_effects.pdf}
  \caption{model output}
  \label{fig:model}
\end{figure}

\section{Discussion}

\section{Conclusions}

Complex measurement stuff. But, still many things to discover

\section{Acknowledgements}

We would like to thank XXXX and XXXX for funding this project. \\

\bibliographystyle{IEEEtran}

\bibliography{my_references.bib}

\end{document}
